\documentclass[10pt]{article}
\usepackage[utf8]{inputenc}
\usepackage{natbib}
\usepackage{url}
\usepackage{hyperref}
\usepackage{graphicx}
\usepackage{geometry}

\geometry{a4paper,top=3cm,bottom=3cm,left=3cm,right=3cm,%
heightrounded,bindingoffset=5mm}

\title{Proposal: Project Computer Vision}
%\author{EROS BIGNARDI, FEDERICO COCCHI, RICCARDO AGAZZOTTI}
\author{
    \textsc{Riccardo Agazzotti} \\[1ex]
    \normalsize \href{mailto:244836@studenti.unimore.it}{244836@studenti.unimore.it}
    \and
    \textsc{Eros Bignardi} \\[1ex]
    \normalsize \href{mailto:289842@studenti.unimore.it}{240696@studenti.unimore.it} 
    \and
    \\[1ex]
    \textsc{Federico Cocchi} \\[1ex]
    \normalsize \href{mailto:289842@studenti.unimore.it}{289842@studenti.unimore.it}
    \\[1ex]
    }
\date{\today}

\begin{document}

\maketitle
\begin{abstract}
    \noindent \textit{Tecniche di Computer Vision a supporto della vendita online di capi di abbigliamento} 
    \newline
    \newline
    \noindent L'e-Commerce crea nuove possibilità e canali di vendita non solo per le grandi aziende, 
    ma soprattutto per le piccole aziende che tramite queste metodologie di vendita possono raggiungere un vasto pubblico situato in tutto il mondo a cui offrire i loro prodotti. 
    La pandemia attuale non ha fatto altro che spingere tale processo. \\
    Da questa situazione il made in italy può trovare un particolare vantaggio per affermarsi anche in nuovi mercati.
    Quanto descritto si applica anche per i prodotti alla persona, quali i capi di abbigliamento, che vengono sempre più venduti online. 
    Per queste tipologie di merce un problema della vendita online è quello di non poter provare il capo, confrontarlo con altri e selezionare correttamente la taglia. \\
    Proprio in quest'ottica il progetto si pone l'obiettivo di facilitare la scelta dei capi a distanza.
    La nostra idea è quella di implementare una pipeline che ci permetta, a partire da un'immagine di una persona, di individuare la posizione ed il tipo di abiti che indossa.
    Per individuare i tipi di abiti è necessario eseguire una loro classificazione dopo aver riconosciuto le specifiche forme dei vari capi. \\
    Dall'immagine di un capo di abbigliamento si può inoltre estrarre un feature vector che rappresenta le sue caratteristiche.
    Da queste utilizzando una misura di similarità si possono individuare altri abiti con caratteristiche comuni da andare a suggerire all'acquirente.
    Tale processo permette di suggerire al cliente alternative al suo outfit.
    %Da ciò si possono suggerire abiti 'vicini' a quelli individuati nella figura iniziale.
    L'ultimo passo della pipeline è quello di utilizzare questi abiti 'vicini' per sostituirli sull'immagine,
    come se la persona li stesse provando in maniera virtuale.
    
    %In un mondo in cui il commercio elettronico è in costante aumento il settore della moda
\end{abstract}
\pagebreak

\section{Pipeline}
 
\subsection{Image Processing}
\begin{itemize}
\item Applicazione di eventuali filtri sull'immagine
\item Segmentazione della persona e dei vari abiti indossati 
\end{itemize}

\subsection{Geometry Algorithm}
\begin{itemize}
\item Trovare le misure corporee della persona per selezionare la taglia corretta
\item Dall'immagine di partenza rendere riconoscibile i singoli capi, anche se la persona che lo indossa assume delle posizioni che non permettono la visione completa del capo
\item Modellazione e trasformazione dei vestiti per permettere di sovrapporli al corpo del soggetto, in base alla posizione assunta, creando l'immagine finale.
\end{itemize}

\subsection{Retrieval Algorithm}
\begin{itemize}
\item \textit{Identificazione parti corpo (busto, gambe e testa)}
\item Identificazione di tutti gli indumenti nella foto e loro classificazione
\item \textit{Identificazione del brand indossato -- template matching}
\end{itemize}

\subsection{Neural Networks and Deep Learning}
\begin{itemize}
\item Classificazione degli elementi principali \textit{(vestito lungo, maglia, pantaloni, scarpe e diversi accessori)}
\item Suggerire e trovare gli elementi simili rispetto a quelli indossati
\item Sviluppare una rete che permetta di far indossare alla persona l'abito finale nell'immagine
\end{itemize}

\subsection{Dataset}
La nostra idea è nata anche dalla possibilità presentata dalla Prof. Cucchiara a lezione, di poter lavorare su un dataset di immagini di abiti e vestiti in possesso dell'università. 
\\ [4ex]

\section{Why we select this project}
Proponiamo questo progetto perchè richiede l'utilizzo delle varie tecniche studiate durante il corso 
ed in secondo luogo perchè risponde ad un fabbisogno concreto che è sempre più sentito, sia da parte dei venditori che dai clienti finali.

\end{document}
